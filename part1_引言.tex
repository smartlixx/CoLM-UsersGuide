\part{引言}
\section{引言}

% 通用陆面模式2024版(The Common Land Model 2024, CoLM 2024)是通用陆面模式的第四个版本(即CLM 1.0, CoLM 2004, CoLM 2014, CoLM 2024)。它是集模式、数据集、性能评估和高性能计算为一体的、自成体系且功能完备的陆面综合模拟研究平台,可广泛应用于数值天气预报/气候预测、水文水资源、生态环境、城市、农林牧等行业的科学研究和精细化业务,适用于多尺度(\textasciitilde 1米至\textasciitilde 100千米)应用。

通用陆面模式2024版(The Common Land Model version 2024, CoLM2024)是集成多物理过程模拟、高分辨率数据集、性能评估与高性能计算的先进陆面模式系统。该模式广泛应用于数值天气预报、气候预测、水文水资源、生态环境、城市及农林牧等领域的科学研究和精细化业务,适用于多尺度(约1米至100公里)的应用。无论是从事天气、气候或地球系统研究,还是关注水文与水资源、生态与环境,以及人类活动的影响与反馈研究,CoLM2024作为一款卓越的陆面过程模式,能够为多领域研究提供有效的研究平台,帮助高效完成相关工作。

CoLM 2024版有以下主要改进或新增,

1. \textbf{新增多种网格和次网格划分方法}。除常用的经纬度网格外,CoLM 2024版还提供了另外两种网格,一种是可根据陆表异质性进行任意加密的非结构网格,另一种是适用于山坡尺度陆面过程模拟的流域单元网格。在网格单元内部,CoLM2024可根据地表覆盖类型、植被功能型或者植物群落对被自然植被覆盖的地表进行进一步的次网格划分。

2. \textbf{使用了更为丰富的基础数据集}。CoLM2024对土地覆盖/土地利用数据、土壤数据、植被结构及属性数据、水文数据、城市数据、作物数据和离线大气驱动数据做了全面的更新或补充。基于新的数据处理和尺度转换方法,在模式中将多源的、具有不同分辨率的数据融合到一起使用。

3. \textbf{改进或新增多个水文和能量过程的参数化方案}。基于新的网格和次网格结构以及基础数据,CoLM2024建立了基于三维植被的辐射传输、湍流、叶片温度和植被截留降水方案,发展了基于物理原理的产流和汇流方案,并新增了积雪内辐射传输、植被水力和生物地球化学等重要过程。

4. \textbf{增加了多个人类活动相关过程}。这些过程包括以三维城市建筑群落为基本结构假设的城市模式、包含多种调度规则的水库模式、土地利用与土地覆盖变化方案以及作物、火灾等模式,为陆面过程与人类活动的相互作用研究提供了有力的工具。

5. \textbf{提升了模式的易操作性和可移植性}。CoLM2024保持了之前版本代码易读易修改的优点,优化了模式配置和数据读写方法,同时针对高分辨率模拟设计了新的并行计算方案。

\subsection{用户指南的目标}

本用户指南旨在帮助不同层次用户学习使用CoLM2024,内容包括模式安装、数据准备、选项配置、运行和调试的详细说明。本指南考虑了初次使用CoLM2024的初级用户,长期固定使用CoLM2024的高级用户和贡献CoLM2024模式模块开发的高级用户各自的需求,设计了分层次学习和参考内容。初级用户可以通过学习用户手册第二部分,从配置一个简单的模式运行实例开始,了解模式能够运行的基本配置、数据和步骤。高级用户可以通过第三部分了解模式的详细配置,并将用户指南作为模式配置的参考手册,根据自己的实际问题,选择恰当的配置方案,包括运行方案,模式的空间单元结构,次网格结构,输出方案,驱动数据和地表数据等。%模式发展用户需要为模式模块的发展深入修改代码并使用github对代码进行版本控制。
模式发展用户需要为模式模块的功能扩展或优化深入修改代码,并通过GitHub平台实施代码版本控制。
模式发展用户若希望对CoLM2024代码的发展做出贡献,可以通过学习第五部分了解如何通过github对用户所发展的代码进行主版本合并,完成沟通、测试等一系列必要流程。
同时,第五部分内容还涉及了CoLM2024特有的自定义数据类型的介绍,为那些需要在CoLM2024模型结构层面上进行代码发展的模式发展用户提供参考。

所有用户均可以通过第2.2节内容了解到如何将CoLM2024配置到一台新机器上,包括编译选项、编译和Netcdf等相关软件需求等。其中,国家超算广州中心“天河星逸”超级计算集群、地球系统数值模拟装置、中山大学大气科学学院摆渡船等机器的配置文件已包含在开源代码内。如果需要对于未测试机器进行移植,须根据第2.2节内容修改已有配置文件,实现新机器CoLM2024模式的正常运行。完成机器配置以后,用户可根据不同需求尝试第四部分模拟实例,检查是否可以根据所提供的重启文件、气象驱动和地表数据,重复实例的模拟结果。同时,用户也可对实例的模拟区域、时间、分辨率、驱动、数据等进行修改,来简易配置自己的模拟。CoLM2024软件需求相对简单,不依赖任何Python环境,模拟配置也相对便捷,较易上手。

本用户指南不仅为初学者提供了详细的入门指导,还为经验丰富的用户提供了更深入的使用建议。无论您是第一次使用CoLM2024,还是需要解决运行过程中遇到问题,本指南都将是您必不可少的入门指南和参考手册。


\subsection{手册结构}

本用户指南第二至第五部分为用户提供从安装到高级配置的完整说明和支持。
本手册第二部分介绍了如何快速地运行一个CoLM2024版模式实例。本部分除简单介绍模型代码结构以外,还可以帮助用户快速配置CoLM2024的运行环境,并编译运行模型。学习本部分是新用户的重要入门步骤。

第三部分详细介绍了所有模式配置细节,包括编译设置、运行设置、输入和输出文件的详细解释。本部分内容是运行CoLM2024的重要参考手册,初级和高级用户在使用CoLM2024的过程中,均可以通过查询相关部分内容,实现模拟的个性化设置。

第四部分包含多个CoLM2024的实际运行案例。本部分内容讲解了如何根据具体的研究题目配置模式,进行多种情形下的陆面过程模拟。用户可以根据兴趣寻求较为接近自身需求的案例,重现案例或根据第三部分内容实现自己的个性化设置。

第五部分包含了模式代码高级介绍和版本控制流程,代码高级介绍包含格点数据类型和并行架构的详细介绍。版本控制流程介绍了模式开发用户参与CoLM2024代码开发的一般步骤。本部分内容对模式中的部分代码做了深入的解释,有助于模式开发者对CoLM2024版进行修改和发展。

\subsection{支持与反馈}

尽管本用户指南已经涵盖了大多数使用案例,但由于每个用户的需求和应用背景可能有所不同,可能会在使用过程中遇到一些特定问题或挑战。我们鼓励通过邮件或在线支持平台与我们联系,反馈问题和建议。我们的技术支持团队将为用户提供及时的帮助,并根据用户反馈持续优化模式和用户指南。

\textbf{联系人及邮件地址:}

张树鹏 <\url{zhangshp8@mail.sysu.edu.cn}>

陆星\hbox{\scalebox{0.6}[1]{吉}\kern-.2em\scalebox{0.6}[1]{力}} <\url{luxingj@mail.sysu.edu.cn}>
%\clearpage
